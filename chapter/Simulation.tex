\chapter{Simulation}
% put these two lines after every \chapter{} command
\vspace{-2em}
\minitoc
To implement and research various FDIR systems on satellites an simulation of satellite dynamics and kinematics is developed. The focus of this thesis is on small satellites and more specifically cubesats. For the simulation of the ADCS of the satellite \cite{auret2012design, JansevanVuuren2015, Jordaan2016} were referenced during the development of the satellite simulation. The simulation was developed in Python to simulate the dynamics and kinematics during a satellite orbit. The faults for the subsystems are also developed within the simulation and will be discussed within this chapter.

\section{Attitude Determination and Control System}

For the mission of the specific satellite in this document the main operational goal of the Attitude Determination and Control System (ADCS) on this specific satellite mission is to control the payload to point towards the centre of the earth. 

\subsection{Coordinate Frames}
The coordinate frames in aerospace is a fundamental part of the ADCS. To determine the orientation and position of an object, it should be relative to a fixed frame. Consequently, the Earth inertial coordinate (EIC) frame is the fixed frame from which every other frame is relative to.

A coordinate frame consists of three orthogonal vectors which is commonly referred to as x, y, and z. The axis of the coordinate frame is appropriately named as X-axis, Y-axis and Z-axis as seen in Figure... A vector (r) within the current coordinate frame can thus be expressed as 
\begin{equation}
\overrightarrow{r} = x\overrightarrow{i} + y\overrightarrow{j} + z\overrightarrow{k}
\end{equation}
where the magnitude of $\overrightarrow{r}$ is denoted as $|\overrightarrow{r}|$ and is equal to 
\begin{equation}
|\overrightarrow{r}| = \sqrt{x^2 + y^2 + z^2}.
\end{equation}

The Earth-centered coordinate frames are dived into two, namely the EIC and earth fixed coordinate (EFC) frame. EFC is fixed to the earth and rotates with it. This frame is important with respect to where the satellite position is with regards to position's on earth, such as the ground station. It is also import for the modelling of the geomagnetic fields. 

% Insert a figure of the earth coordinate frames here

The EIC is defined as the Z-axis pointing towards the north pole, the X-axis pointing towards the Vernal Equinox, $\Upsilon$, and the Y-axis completing the orthogonal set. The EFC is a copy of the EIC, with the Z-axis being identical, however the EFC rotates with the earth. The EFC in relation to the EIC can be expressed by a single angle of rotation, which is the Greenwich Hour Angle (GHA), $\alpha_G$. With the knowledge of $t$ --- the elapsed time since $t_0$, $w_E$ --- the angular rate of the earth, and $\alpha_{G,0}$ --- the GHA at $t = t_0$, $\alpha_G$ can be calculated as 
\begin{equation}
\alpha_G = w_Et + \alpha_{G,0}
\end{equation}
To transform a vector from one coordinate frame to another, a transformation matrix, $\boldsymbol{A}$, is required. For example vector $\overrightarrow{r}_{EFC}$ can be transformed to $\overrightarrow{r}_{EIC}$ with 
\begin{equation}
\overrightarrow{r}_{EIC} = \boldsymbol{A}^{EIC}_{EFC}\overrightarrow{r}_{EFC}
\end{equation}
with $\boldsymbol{A}^{EIC}_{EFC}$ being the EFC-to-EIC transformation matrix. Due to the definition of both coordinate frames, $\boldsymbol{A}^{EIC}_{EFC}$ can be defined .

\begin{equation}
\boldsymbol{A}^{EIC}_{EFC} = 
\begin{bmatrix}
	cos(\alpha_G) & -sin(\alpha_G) & 0\\
	sin(\alpha_G) & cos(\alpha_G) & 0 \\
	0 & 0 & 1
\end{bmatrix}
\end{equation}

To determine the satellite position, satellite coordinate frames must be used. Three satellite-centred coordinate frames are used, namely the inertial-reference coordinate frame (the satellite does not rotate), the orbit-referenced coordinate (ORC) frame and the satellite body coordinate (SBC) frame. The IRC frame is only acknowledged, since it is the frame that is fixed (as it does not rotate around the centre of the satellite), however it changes position with the orbit of the satellite. This frame is not used to determine the position of the satellite and will not be referenced for the remainder of this document.

The ORC frame changes location as the satellite moves, however the Z-axis is always pointing towards the centre of the earth, with the Y-axis being the anti-normal and the X-axis completing the orthogonal set. To transform a vector from the EIC frame to the ORC frame the unit position vector, $\overrightarrow{r}_{sat}$ and the unit velocity vector, $\overrightarrow{v}_{sat}$ in EIC \cite{Chen_ground-target}.

\begin{equation}
	\boldsymbol{A}^{ORC}_{EIC} = 
	\begin{bmatrix}
		\hat{u} & \hat{v} & \hat{w}\\
	\end{bmatrix}^T
\end{equation}
where
\begin{equation}
\hat{w} = -\frac{r_{sat}}{||r_{sat}||}
\end{equation}
\begin{equation}
\hat{v} = -\frac{r_{sat} \times v_{sat}}{||r_{sat} \times v_{sat}||}
\end{equation}
\begin{equation}
\hat{u} = \hat{v} \times \hat{w}
\end{equation}

The SBC frame is the frame fixed to the satellite and it is the relative rotation of the satellite in relation to the ORC. Thus for the mission of this satellite it is required that the SBC and ORC frames coincide. For the transformation of a vector from the ORC to SBC frame, the direct cosine matrix (DCM) also referred to as $\boldsymbol{A}$ or $\boldsymbol{A}^{SBC}_{ORC}$ is used. For the remainder of the document the DCM will be referred to as $\boldsymbol{A}^{SBC}_{ORC}$ to avoid any confusion. The calculation of this transformation matrix will be discussed in $\S$\ref{subsection_quaternions}.

% Insert a figure of the satellite coordinate frames here

\subsection{Attitude}
\label{subsection_quaternions}
To determine the attitude of an object, a model must be used to determine the rotation of an object in three dimensions. For this the visual and intuitive example of the Euler angles exist. Euler angles are the rotation of an object around three orthogonal axis, that change orientation with the rotation of the object. The three axes, denoted by X, Y and Z rotate with the object as depicted in Figure~\ref{Figure-Euler_angles}.

\begin{figure}[h!tb]
\label{Figure-Euler_angles}
\begin{tikzpicture}[scale=2.5,tdplot_main_coords]
	%\tdplotsetmaincoords{70}{110}
	% Set origin of main (body) coordinate system
	\coordinate (O) at (0,0,0);
	
	% Draw main coordinate system
	\draw[red, ,->] (0,0,0) -- (1,0,0) node[anchor=north east]{$x_{\mathcal{I}}$};
	\draw[red, ,->] (0,0,0) -- (0,1,0) node[anchor=north west]{$y_{\mathcal{I}}$};
	\draw[red, ,->] (0,0,0) -- (0,0,1) node[anchor=south]{$z_{\mathcal{I}}$};
	
	
	
	% Intermediate frame 1
	\tdplotsetrotatedcoords{-60}{0}{0}
	\draw[tdplot_rotated_coords,->, blue] (0,0,0) -- (1,0,0) node[anchor=north east]{$x'$};
	\draw[tdplot_rotated_coords,->, blue] (0,0,0) -- (0,1,0) node[anchor=west]{$y'$};
	\draw[tdplot_rotated_coords,->, blue] (0,0,0) -- (0,0,1) node[anchor=west]{$z'$};
	
	\tdplotsetrotatedthetaplanecoords{90}
	%draw theta arc and label
	%\tdplotdrawarc[tdplot_rotated_coords,->,color=blue]{(0,0,0)}{0.5}{0}{350}{anchor=south west,color=gray}{$\alpha$}
	\tdplotdrawarc[tdplot_rotated_coords,->,color=gray]{(0,0,0)}{0.5}{80}{90}{anchor=south west,color=gray}{$\alpha$}
	\tdplotdrawarc[tdplot_rotated_coords,->,color=gray]{(0,0,0)}{0.5}{170}{180}{anchor=south west,color=gray,  yshift = -15 pt}{$\alpha$}
	
	%% Intermediate frame 2
	\tdplotsetrotatedcoords{-60}{15}{0}
	\draw[,tdplot_rotated_coords,->, green] (0,0,0) -- (1,0,0) node[anchor=
	north]{};
	\draw[,tdplot_rotated_coords,->, green] (0,0,0) -- (0,1,0)
	node[anchor=west]{$y''$};
	\draw[,tdplot_rotated_coords,->, green] (0,0,0) -- (0,0,1)
	node[anchor=south]{$z''$};
	\tdplotsetrotatedthetaplanecoords{60}
	%draw theta arc and label
	%\tdplotdrawarc[tdplot_rotated_coords,->,color=green]{(0,0,0)}{0.5}{0}{350}{anchor=north,color=gray}{$\beta$}
	\tdplotdrawarc[tdplot_rotated_coords,->,color=gray]{(0,0,0)}{0.5}{80}{90}{anchor=north,color=gray}{$\beta$}
	\tdplotdrawarc[tdplot_rotated_coords,->,color=gray]{(0,0,0)}{0.5}{310}{320}{anchor=south west,color=gray}{$\beta$}
	% 
	% Rotate to final frame
	\tdplotsetrotatedcoords{-60}{15}{45}
	\draw[thick,tdplot_rotated_coords,->, cyan] (0,0,0) -- (1,0,0)
	node[anchor=west]{$x_{\mathcal{B}}$, \textcolor{green}{$x''$}};
	\draw[thick,tdplot_rotated_coords,->, cyan] (0,0,0) -- (0,1,0) node[anchor=west]{$y_{\mathcal{B}}$};
	\draw[thick,tdplot_rotated_coords,->, cyan] (0,0,0) -- (0,0,1) node[anchor=south]{$z_{\mathcal{B}}$};
	
	\tdplotsetrotatedthetaplanecoords{30}
	%draw theta arc and label
	%\tdplotdrawarc[tdplot_rotated_coords,->,color=cyan]{(0,0,0)}{0.5}{0}{350}{anchor=north,color=gray}{$\gamma$}
	\tdplotdrawarc[tdplot_rotated_coords,->,color=gray]{(0,0,0)}{0.5}{215}{225}{anchor=north,color=gray, yshift = 15pt}{$\gamma$}
	\tdplotdrawarc[tdplot_rotated_coords,->,color=gray]{(0,0,0)}{0.5}{328}{338}{anchor=south west,color=gray}{$\gamma$}
	
\end{tikzpicture}
\end{figure}


Euler angles and DCM will be acknowledged and explained why they are not applicable.
In some literature the first and last quaternion are swapped.

\subsection{Satellite Kinematics and Dynamics}

\subsection{Rungka-kutta}

\section{Environment}
\subsection{Earth Orbit}
Earth orbit according to sgp4 and also the placement of the earth sensor.

Show plot of 3D earth orbit...

\subsection{Sun}
The calculations for the sun position and also the placement of the coarse and fine sun sensor.

Show graph of sun plot...

\subsection{Geomagnetic field}

\begin{equation}
\label{Eq-Geomagnetic_field}
V(r_s,\theta, \lambda) = R_E \sum_{n=1}^{k}\left(\frac{R_E}{r_s}^{n+1}\right)\sum_{m=0}^{n}\left(g_n^mcos(m\lambda) + h_n^msin(m\lambda)\right)P_n^m(\theta)
\end{equation}

Show graph of geomagnetic plot...

\section{Sensor models}
\subsection{Position of Sensors and Field of View}

\subsection{Noise}

\section{Disturbance models}
\subsection{Gravity Gradient}

\subsection{Aerodynamic Disturbance}
\cite{Steyn2014}

\subsection{Wheel Imbalance}

\section{Attitude Determination}
In this section discuss the Kalman filter.

\subsection{Kalman Filter}
\cite{Jones2017}

\section{Attitude Control}
Magnetic control during detumbling
Reaction wheel control during normal operation

\section{Typical Faults}
For the simulation of the satellite and the induced faults to train and test various anomaly detection methodologies a database of typical faults is required. \textcite{tafazoli2009study} made a study of the percentage of failure per subsystem. 

\subsection{Probability of Fault Occurence}
The occurrence of a fault depends on the reliability of that equipment. \textcite{Guo2014} studied the reliability of small satellites and calculated the parameters for the Weibull distribution based on real data. To model the probability of a fault to occur the probability density function is used \cite{Jones2017}.

This probability however is small and for the training of the system the data is too sparse for the computational abilities of any regular PC. Thus the probability of a failure during training is fixed to $1/1000000$ to produce the data required for the anomaly detection with a million test samples.

\subsection{Set of faults}
A set of typical faults for the ADCS is shown in Table~\ref{ADCS fault table}. 

\newpage
\begin{sidewaystable}[]
	\label{ADCS fault table}
	\begin{tabular}{|l|c|l|l|l|l|}
		\hline
		\multicolumn{6}{|c|}{\textbf{Internal Faults}} \\ \hline
		\textbf{Fault classes} &
		\multicolumn{1}{l|}{\textbf{\begin{tabular}[c]{@{}l@{}}Failure rate \\ per hour\end{tabular}}} &
		\textbf{Fault causes} &
		\textbf{References} &
		\textbf{Possible effect} &
		\textbf{Possible permutations} \\ \hline
		\multirow{4}{*}{Reaction wheels} &
		\multicolumn{1}{l|}{\multirow{4}{*}{2.5E-7 \cite{Spilhaus1987}}} &
		\begin{tabular}[c]{@{}l@{}}Reaction wheel electronics \\ fail\end{tabular} &
		\cite{allen2012satellite} \cite{Jacklin2019} &
		\begin{tabular}[c]{@{}l@{}}Does not respond \\ to control inputs\end{tabular} &
		\begin{tabular}[c]{@{}l@{}}Momentum remains the same \\ or decreases slightly due to \\ friction\end{tabular} \\ \cline{3-6} 
		&
		\multicolumn{1}{l|}{} &
		Overheated reaction wheel &
		\cite{Wintoft} &
		Decrease in speed &
		1\% of initial speed per second \\ \cline{3-6} 
		&
		\multicolumn{1}{l|}{} &
		\begin{tabular}[c]{@{}l@{}}Catastrophic failure (cause \\ unknown)\end{tabular} &
		\cite{Choi2011} &
		Stops rotating &
		0 \\ \cline{3-6} 
		&
		\multicolumn{1}{l|}{} &
		\begin{tabular}[c]{@{}l@{}}Increase in rotation speed \\ (Unknown cause)\end{tabular} &
		\begin{tabular}[c]{@{}l@{}}Gerhard Janse \\ van Vuuren\end{tabular} &
		\begin{tabular}[c]{@{}l@{}}Wheel speed \\ increases\end{tabular} &
		\begin{tabular}[c]{@{}l@{}}Between 90-100\% of maximum \\ wheel speed\end{tabular} \\ \hline
		Magnetorquers &
		\multicolumn{1}{l|}{ADCS fault table8.15E-9 \cite{Spilhaus1987}} &
		Polarities are inverted &
		\cite{Crowell2011} &
		Incorrect rotation &
		\\ \hline
		\multirow{2}{*}{Magnetometers} &
		\multicolumn{1}{l|}{\multirow{2}{*}{8.15E-9 \cite{Spilhaus1987}}} &
		Unknown &
		\begin{tabular}[c]{@{}l@{}}Gerhard Janse \\ van Vuuren\end{tabular} &
		Stops reacting &
		\begin{tabular}[c]{@{}l@{}}Provides no feedback or the \\ output remains constant\end{tabular} \\ \cline{3-6} 
		&
		\multicolumn{1}{l|}{} &
		\begin{tabular}[c]{@{}l@{}}Magnetometers and magne-\\ torquers interfered with \\ each other\end{tabular} &
		\cite{Jacklin2019} &
		\begin{tabular}[c]{@{}l@{}}Noise on magneto-\\ meters and noise \\ on control of mag-\\ netorquers\end{tabular} &
		\begin{tabular}[c]{@{}l@{}}Between x3 and x5 times the \\ normal noise magnitude \\ Guassian distribution\end{tabular} \\ \hline
		Earth Sensor &
		- &
		Unknown &
		\cite{Robertson2019} &
		\begin{tabular}[c]{@{}l@{}}Noisy Earth Sensor \\ effected pointing \\ accuracy\end{tabular} &
		\begin{tabular}[c]{@{}l@{}}Between x5 and x10 times the \\ normal sensor noise based on \\ Guassian distribution\end{tabular} \\ \hline
		\multirow{2}{*}{Sun sensor} &
		\multirow{2}{*}{-} &
		\begin{tabular}[c]{@{}l@{}}Cross-wired during instal-\\ lation\end{tabular} &
		\cite{Crowell2011} &
		\begin{tabular}[c]{@{}l@{}}Erroneous \\ measurements\end{tabular} &
		Uniform random values \\ \cline{3-6} 
		&
		&
		Unknown &
		\cite{Jacklin2019} &
		Sun sensor fails &
		output is 0 \\ \hline
		Star tracker &
		- &
		\begin{tabular}[c]{@{}l@{}}Shutter on star tracker is \\ closed\end{tabular} &
		\cite{Crowell2011} &
		Star tracker fails &
		output is 0 \\ \hline
		Overall control &
		- &
		\begin{tabular}[c]{@{}l@{}}Incorrect control law or \\ variation \\ thereof\end{tabular} &
		\begin{tabular}[c]{@{}l@{}}Gerhard Janse \\ van Vuuren\end{tabular} &
		\begin{tabular}[c]{@{}l@{}}Angular velocity \\ suddenly increases \\ or decreases or \\ oscillation results\end{tabular} &
		\begin{tabular}[c]{@{}l@{}}Increase to 75 - 100\% \\ Decrease to 0 - 25\%\\ Oscillates\end{tabular} \\ \hline
		\multirow{3}{*}{\begin{tabular}[c]{@{}l@{}}Common data \\ transmission errors\end{tabular}} &
		\multirow{3}{*}{-} &
		Sign flip &
		\cite{Crowell2011} &
		Processor-based &
		\begin{tabular}[c]{@{}l@{}}Processor outputs and/or \\ inputs experience a sign flip\end{tabular} \\ \cline{3-6} 
		&
		&
		Bit flip &
		N/A &
		&
		\begin{tabular}[c]{@{}l@{}}Processor outputs and/or \\ inputs experience a bit flip flip\end{tabular} \\ \cline{3-6} 
		&
		&
		Insertion of zeros &
		\cite{Jacklin2019} &
		&
		\begin{tabular}[c]{@{}l@{}}Processor outputs and/or inputs \\ experience an insertion of a zero\end{tabular} \\ \hline
		\begin{tabular}[c]{@{}l@{}}Possible sensors \\ errors\end{tabular} &
		- &
		Unknown &
		N/A &
		High sensor noise &
		\begin{tabular}[c]{@{}l@{}}Between x5 and x10 times the \\ normal sensor noise based on \\ Guassian distribution\end{tabular} \\ \hline
	\end{tabular}
\end{sidewaystable}

\newpage