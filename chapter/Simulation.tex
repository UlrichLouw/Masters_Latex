\chapter{Simulation}
% put these two lines after every \chapter{} command
\vspace{-2em}
\minitoc
To implement and research various FDIR systems on satellites an simulation of satellite dynamics and kinematics is developed. The focus of this thesis is on small satellites and more specifically cubesats. For the simulation of the ADCS of the satellite \cite{auret2012design, JansevanVuuren2015, Jordaan2016} were referenced during the development of the satellite simulation. The simulation was developed in Python to simulate the dynamics and kinematics during a satellite orbit. The faults for the subsystems are also developed within the simulation and will be discussed within this chapter.

\section{ADCS}

\subsection{Coordinate Frames}


The main operational goal of the ADCS on this specific satellite mission is to control the payload to point towards the centre of the earth. 

\subsection{Euler Angles and Quaternions}

\subsection{Satellite Kinematics and Dynamics}

\subsection{Rungka-kutta}

\section{Environment}
\subsection{Earth Orbit}
Earth orbit according to sgp4 and also the placement of the earth sensor.

Show plot of 3D earth orbit...

\subsection{Sun}
The calculations for the sun position and also the placement of the coarse and fine sun sensor.

\subsection{Geomagnetic field}

\begin{equation}
V(r_s,\theta, \lambda) = R_E \sum_{n=1}^{k}\left(\frac{R_E}{r_s}^{n+1}\right)\sum_{m=0}^{n}\left(g_n^mcos(m\lambda) + h_n^msin(m\lambda)\right)P_n^m(\theta)
\end{equation}

Show graph of geomagnetic plot...

\section{Sensor models}
\subsection{Position of Sensors and Field of View}

\begin{figure}
\centering
\begin{tikzpicture}[tdplot_main_coords]
		% Start of cone
		\coordinate (O) at (0,0,0.5);
		
		%% make sure to draw everything from back to front
		%\coneback[surface]{-1.5}{2.5}{-15}
		%\coneback[surface]{-3}{2}{-10}
		\draw (0,0,-5) -- (O);
		%\conefront[surface]{-3}{2}{-10}
		%\conefront[surface]{-1.5}{2.5}{-15}
		%\filldraw[surface] circle (3);
		\draw[->] (-6,0,0) -- (6,0,0) node[right] {$x$};
		\draw[->] (0,-6,0) -- (0,6,0) node[right] {$y$};
		
		%\coneback[surface]{3}{2}{10}
		\draw[->] (O) -- (0,0,5) node[above] {$z$};

	%%% Edit the following coordinate to change the shape of your
%%% cuboid

%% Vanishing points for perspective handling
\coordinate (P1) at (-7cm,1.5cm); % left vanishing point (To pick)
\coordinate (P2) at (8cm,1.5cm); % right vanishing point (To pick)

%% (A1) and (A2) defines the 2 central points of the cuboid
\coordinate (A1) at (0em,0cm); % central top point (To pick)
\coordinate (A2) at (0em,-2cm); % central bottom point (To pick)

%% (A3) to (A8) are computed given a unique parameter (or 2) .8
% You can vary .8 from 0 to 1 to change perspective on left side
\coordinate (A3) at ($(P1)!.8!(A2)$); % To pick for perspective 
\coordinate (A4) at ($(P1)!.8!(A1)$);

% You can vary .8 from 0 to 1 to change perspective on right side
\coordinate (A7) at ($(P2)!.7!(A2)$);
\coordinate (A8) at ($(P2)!.7!(A1)$);

%% Automatically compute the last 2 points with intersections
\coordinate (A5) at
(intersection cs: first line={(A8) -- (P1)},
second line={(A4) -- (P2)});
\coordinate (A6) at
(intersection cs: first line={(A7) -- (P1)}, 
second line={(A3) -- (P2)});

%%% Depending of what you want to display, you can comment/edit
%%% the following lines

%% Possibly draw back faces

\fill[gray!90] (A2) -- (A3) -- (A6) -- (A7) -- cycle; % face 6
\node at (barycentric cs:A2=1,A3=1,A6=1,A7=1) {\tiny f6};

\fill[gray!50] (A3) -- (A4) -- (A5) -- (A6) -- cycle; % face 3
\node at (barycentric cs:A3=1,A4=1,A5=1,A6=1) {\tiny f3};

\fill[gray!30] (A5) -- (A6) -- (A7) -- (A8) -- cycle; % face 4
\node at (barycentric cs:A5=1,A6=1,A7=1,A8=1) {\tiny f4};

\draw[thick,dashed] (A5) -- (A6);
\draw[thick,dashed] (A3) -- (A6);
\draw[thick,dashed] (A7) -- (A6);

%% Possibly draw front faces

% \fill[orange] (A1) -- (A8) -- (A7) -- (A2) -- cycle; % face 1
% \node at (barycentric cs:A1=1,A8=1,A7=1,A2=1) {\tiny f1};
\fill[gray!50,opacity=0.2] (A1) -- (A2) -- (A3) -- (A4) -- cycle; % f2
\node at (barycentric cs:A1=1,A2=1,A3=1,A4=1) {\tiny f2};
\fill[gray!90,opacity=0.2] (A1) -- (A4) -- (A5) -- (A8) -- cycle; % f5
\node at (barycentric cs:A1=1,A4=1,A5=1,A8=1) {\tiny f5};

%% Possibly draw front lines
\draw[thick] (A1) -- (A2);
\draw[thick] (A3) -- (A4);
\draw[thick] (A7) -- (A8);
\draw[thick] (A1) -- (A4);
\draw[thick] (A1) -- (A8);
\draw[thick] (A2) -- (A3);
\draw[thick] (A2) -- (A7);
\draw[thick] (A4) -- (A5);
\draw[thick] (A8) -- (A5);

% Possibly draw points
% (it can help you understand the cuboid structure)
\foreach \i in {1,2,...,8}
{
	\draw[fill=black] (A\i) circle (0.15em)
	node[above right] {\tiny \i};
}
		\coneback[surface]{1.5}{2.5}{15}
		\conefront[surface]{1.5}{2.5}{15}
\end{tikzpicture}
\end{figure}

\tdplotsetmaincoords{70}{120}
\begin{figure}
\centering
\begin{tikzpicture}[tdplot_main_coords]
\def\BigSide{5}
\def\SmallSide{1.5}
\pgfmathsetmacro{\CalcSide}{\BigSide-\SmallSide}

% The vertex at V
\tdplotsetcoord{P}{sqrt(3)*\BigSide}{55}{45}

\coordinate (sxl) at (\BigSide,\CalcSide,\BigSide);
\coordinate (syl) at (\CalcSide,\CalcSide,\BigSide);
\coordinate (szl) at (\CalcSide,\BigSide,\BigSide);

\draw[dashed] 
(0,0,0) -- (Px)
(0,0,0) -- (Py)
(0,0,0) -- (Pz);
\draw[->] 
(Px) -- ++ (1,0,0) node[anchor=north east]{$x$};
\draw[->]
(Py) -- ++(0,1,0) node[anchor=north west]{$y$};
\draw[->] 
(Pz) -- ++(0,0,1) node[anchor=south]{$z$};

\draw[thick]
(Pxz) -- (P) -- (Pxy) -- (Px) -- (Pxz) -- (Pz) -- (Pyz) -- (P); 
\draw[thick]
(Pyz) -- (Py) -- (Pxy);


\end{tikzpicture}
\end{figure}


\section{Disturbance models}
\subsection{Gravity Gradient}

\subsection{Aerodynamic Disturbance}

\subsection{Wheel Imbalance}

\section{Attitude Determination}
IN this chapter discuss the Kalman filter.

\section{Attitude Control}
Magnetic control during detumbling
Reaction wheel control during normal operation

\section{Typical Faults}
For the simulation of the satellite and the induced faults to train and test various anomaly detection methodologies a database of typical faults is required. \textcite{tafazoli2009study} made a study of the percentage of failure per subsystem. 

\subsubsection{Faults}
The occurrence of a fault depends on the reliability of that equipment. \textcite{Guo2014} studied the reliability of small satellites and calculated the parameters for the Weibull distribution based on real data. A set of typical faults for the ADCS is shown in Table~\ref{ADCS fault table}. 

\newpage
\begin{sidewaystable}[]
	\label{ADCS faults}
	\begin{tabular}{|l|c|l|l|l|l|}
		\hline
		\multicolumn{6}{|c|}{\textbf{Internal Faults}} \\ \hline
		\textbf{Fault classes} &
		\multicolumn{1}{l|}{\textbf{\begin{tabular}[c]{@{}l@{}}Failure rate \\ per hour\end{tabular}}} &
		\textbf{Fault causes} &
		\textbf{References} &
		\textbf{Possible effect} &
		\textbf{Possible permutations} \\ \hline
		\multirow{4}{*}{Reaction wheels} &
		\multicolumn{1}{l|}{\multirow{4}{*}{2.5E-7 \cite{Spilhaus1987}}} &
		\begin{tabular}[c]{@{}l@{}}Reaction wheel electronics \\ fail\end{tabular} &
		\cite{allen2012satellite} \cite{Jacklin2019} &
		\begin{tabular}[c]{@{}l@{}}Does not respond \\ to control inputs\end{tabular} &
		\begin{tabular}[c]{@{}l@{}}Momentum remains the same \\ or decreases slightly due to \\ friction\end{tabular} \\ \cline{3-6} 
		&
		\multicolumn{1}{l|}{} &
		Overheated reaction wheel &
		\cite{Wintoft} &
		Decrease in speed &
		1\% of initial speed per second \\ \cline{3-6} 
		&
		\multicolumn{1}{l|}{} &
		\begin{tabular}[c]{@{}l@{}}Catastrophic failure (cause \\ unknown)\end{tabular} &
		\cite{Choi2011} &
		Stops rotating &
		0 \\ \cline{3-6} 
		&
		\multicolumn{1}{l|}{} &
		\begin{tabular}[c]{@{}l@{}}Increase in rotation speed \\ (Unknown cause)\end{tabular} &
		\begin{tabular}[c]{@{}l@{}}Gerhard Janse \\ van Vuuren\end{tabular} &
		\begin{tabular}[c]{@{}l@{}}Wheel speed \\ increases\end{tabular} &
		\begin{tabular}[c]{@{}l@{}}Between 90-100\% of maximum \\ wheel speed\end{tabular} \\ \hline
		Magnetorquers &
		\multicolumn{1}{l|}{8.15E-9 \cite{Spilhaus1987}} &
		Polarities are inverted &
		\cite{Crowell2011} &
		Incorrect rotation &
		\\ \hline
		\multirow{2}{*}{Magnetometers} &
		\multicolumn{1}{l|}{\multirow{2}{*}{8.15E-9 \cite{Spilhaus1987}}} &
		Unknown &
		\begin{tabular}[c]{@{}l@{}}Gerhard Janse \\ van Vuuren\end{tabular} &
		Stops reacting &
		\begin{tabular}[c]{@{}l@{}}Provides no feedback or the \\ output remains constant\end{tabular} \\ \cline{3-6} 
		&
		\multicolumn{1}{l|}{} &
		\begin{tabular}[c]{@{}l@{}}Magnetometers and magne-\\ torquers interfered with \\ each other\end{tabular} &
		\cite{Jacklin2019} &
		\begin{tabular}[c]{@{}l@{}}Noise on magneto-\\ meters and noise \\ on control of mag-\\ netorquers\end{tabular} &
		\begin{tabular}[c]{@{}l@{}}Between x3 and x5 times the \\ normal noise magnitude \\ Guassian distribution\end{tabular} \\ \hline
		Earth Sensor &
		- &
		Unknown &
		\cite{Robertson2019} &
		\begin{tabular}[c]{@{}l@{}}Noisy Earth Sensor \\ effected pointing \\ accuracy\end{tabular} &
		\begin{tabular}[c]{@{}l@{}}Between x5 and x10 times the \\ normal sensor noise based on \\ Guassian distribution\end{tabular} \\ \hline
		\multirow{2}{*}{Sun sensor} &
		\multirow{2}{*}{-} &
		\begin{tabular}[c]{@{}l@{}}Cross-wired during instal-\\ lation\end{tabular} &
		\cite{Crowell2011} &
		\begin{tabular}[c]{@{}l@{}}Erroneous \\ measurements\end{tabular} &
		Uniform random values \\ \cline{3-6} 
		&
		&
		Unknown &
		\cite{Jacklin2019} &
		Sun sensor fails &
		output is 0 \\ \hline
		Star tracker &
		- &
		\begin{tabular}[c]{@{}l@{}}Shutter on star tracker is \\ closed\end{tabular} &
		\cite{Crowell2011} &
		Star tracker fails &
		output is 0 \\ \hline
		Overall control &
		- &
		\begin{tabular}[c]{@{}l@{}}Incorrect control law or \\ variation \\ thereof\end{tabular} &
		\begin{tabular}[c]{@{}l@{}}Gerhard Janse \\ van Vuuren\end{tabular} &
		\begin{tabular}[c]{@{}l@{}}Angular velocity \\ suddenly increases \\ or decreases or \\ oscillation results\end{tabular} &
		\begin{tabular}[c]{@{}l@{}}Increase to 75 - 100\% \\ Decrease to 0 - 25\%\\ Oscillates\end{tabular} \\ \hline
		\multirow{3}{*}{\begin{tabular}[c]{@{}l@{}}Common data \\ transmission errors\end{tabular}} &
		\multirow{3}{*}{-} &
		Sign flip &
		\cite{Crowell2011} &
		Processor-based &
		\begin{tabular}[c]{@{}l@{}}Processor outputs and/or \\ inputs experience a sign flip\end{tabular} \\ \cline{3-6} 
		&
		&
		Bit flip &
		N/A &
		&
		\begin{tabular}[c]{@{}l@{}}Processor outputs and/or \\ inputs experience a bit flip flip\end{tabular} \\ \cline{3-6} 
		&
		&
		Insertion of zeros &
		\cite{Jacklin2019} &
		&
		\begin{tabular}[c]{@{}l@{}}Processor outputs and/or inputs \\ experience an insertion of a zero\end{tabular} \\ \hline
		\begin{tabular}[c]{@{}l@{}}Possible sensors \\ errors\end{tabular} &
		- &
		Unknown &
		N/A &
		High sensor noise &
		\begin{tabular}[c]{@{}l@{}}Between x5 and x10 times the \\ normal sensor noise based on \\ Guassian distribution\end{tabular} \\ \hline
	\end{tabular}
	\label{ADCS fault table}
\end{sidewaystable}

\newpage