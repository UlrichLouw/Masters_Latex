%%%%%%%%%%%%%%%%%%%%%%%%%%%%%%%%%%%%%%%%%%%%%%%%
%
% start writing
%
%%%%%%%%%%%%%%%%%%%%%%%%%%%%%%%%%%%%%%%%%%%%%%%%


\chapter{Introduction}
% put these two lines after every \chapter{} command
\vspace{-2em}
\minitoc

\startarabicpagenumbering % must be just after the first \chapter{} command


\blindtext

\section{Background}

\blindtext

\blindtext

\section{Informal problem description}

\blindtext

\blindtext

\section{Research hypothesis}

\Blindtext

\section{Scope and objectives}

The following objectives will be pursued in this project/thesis/dissertation:
\begin{enumerate}[label=\Roman*]										% \usepackage{enumitem}
 \item To \textit{conduct} a thorough survey of the literature related to:
 \begin{enumerate}[label=(\alph*)]
  \item facility location problems in general,
  \item models for the placement of a network of radio transmitters in particular,
  \item the nature of parameters required to describe effective radio transmission, and
  \item terrain elevation data required to generate an instance of the bi-objective radio transmitter location problem described in the previous section.
 \end{enumerate}
 \item  To \textit{establish} an suitable framework for evaluating the effectiveness of a given set of placement locations for a network of radio transmitters in respect of its total area coverage and its mutual area coverage.
 \item To \textit{formulate} a bi-objective facility location model suitable as a basis for decision support in respect of the location of a network of radio transmitters with a view to identify high-quality trade-offs between maximising total coverage area and maximising mutual coverage area.  The model should take as input the parameters and data identified in Objective~I(c)--(d) and function within the context of the framework of Objective~II.
 \item To \textit{design} a generic \textit{decision support system} (DSS) capable of suggesting high-quality trade-off locations for user-specified instances of the bi-objective radio transmitter location problem described in the previous section.  This DSS should incorporate the location model of Objective~III.
 \item To \textit{implement} a concept demonstrator of the DSS of Objective IV in an applicable software platform.  This DSS should be flexible in the sense of being able to take as input an instance of the bi-objective radio transmitter location problem described in the previous section via user-specification of the parameters and data of Objectives I(c)--(d) and produce as output a set of high-quality trade-off transmitter locations for that instance.
 \item To \textit{verify} and validate the implementation of Objective V according to generally accepted modelling guidelines.
 \item To \textit{apply} the concept demonstrator of Objective V to a special case study involving realistic radio transmission parameters and real elevation data for a specified portion of terrain.
 \item To \textit{evaluate} the effectiveness of the DSS and associated concept demonstrator of Objectives~IV--VI in terms of its capability to identify a set of high-quality trade-off solutions for a network of radio transmitter locations.
 \item To \textit{recommend} sensible follow-up work related to the work in this project which may be pursued in future.
\end{enumerate}

\section{Research methodology}
\blindtext

\section{Project/thesis/dissertation organisation}
\blindtext